\documentclass{article}

\usepackage[utf8]{inputenc}
%\usepackage[T1]{fontenc}
\usepackage{imakeidx}
\usepackage[spanish]{babel}
\usepackage{enumitem} % para utilizar [noitemsep, nolistsep]
\setlist[enumerate]{label*=\arabic*.} % te crea automaticamente sub enumeraciones
\setlength{\parindent}{2em} 
\setlength{\parskip}{1em} % para el espacio entre párrafos
\renewcommand{\baselinestretch}{1.1} % para el espacio entre las líneas
\usepackage{geometry} % paquete para acomodar márgenes
\geometry{a4paper, total={150mm,257mm}, left=30mm, top=20mm,}
\usepackage{schemata}
\newcommand\diagram[2]{\schema{\schemabox{#1}}{\schemabox{#2}}}

\addto{\captionsspanish}{\renewcommand*{\contentsname}{}} % Lo que este entre {} es lo que se escribira como titulo de la table of contents

\begin{document}

\begin{center}
    \LARGE\textbf{{Apéndice}}
\end{center}

\tableofcontents
  
\newpage

\section{Textos de unidad 1}

\subsection{``Exposición sobre los libros del arte de la Lógica''}

\begin{center}
    \large{\textbf{Guillermo de Ockham} [1285? - 1350]} 
\end{center}
    Fue un fraile franciscano, filósofo y lógico escolástico inglés, oriundo de Ockham. Como miembro de la Orden Franciscana dedicó la vida a la pobreza extrema. Murió a causa de la peste negra. Se le conoce principalmente por la Navaja de Ockham, un principio metodológico e innovador, y por sus obras significativas en lógica, medicina y teología. \par
    
\begin{center}
    \large{\textbf{Proemio}}
\end{center}
    
    ``Ya que todo el que hace algo necesita de algo que le dirija, porque puede errar en sus obras y actividades, y el entendimiento humano procede necesariamente en la adquisición de la ciencia y en su perfeccionamiento, de lo desconocido a lo conocido, en lo cual eso que le dirige puede errar por mucho, se ha hecho necesario encontrar algún arte con ayuda de la cual pudiese conocer con evidencia los razonamientos verdaderos y falsos, para poder discernir con certeza la verdad de la falsedad. Este arte es la Lógica, por cuya ignorancia, según dice Aristóteles en su ``Física" (Libro I), muchos entre los antiguos cayeron en diversos errores. \par
    
    ...Ahora conviene tratar de la utilidad de la Lógica. Son muchas sus utilidades: Una de ellas es la facilidad en discernir lo verdadero de lo falso. En efecto, cuando la Lógica se posee a perfección, se juzga con facilidad qué es verdad y qué es falso, y esto se puede llegar a saber por medio de las proposiciones evidentes de por sí. Pues, como en esas proposiciones lo único que hay que hacer es proceder ordenadamente de las proposiciones evidentes de por sí a las últimas consecuencias que se derivan de ellas, y es la Lógica la que enseña tal modo de discurrir y tal proceso, se sigue que con su ayuda se encuentra fácilmente la verdad en tales cosas, y por la misma razón se discierne con toda facilidad lo verdadero de lo falso. \par
    
    La segunda utilidad es la prontitud o rapidez en el responder. En este arte se enseña qué es una contradicción, qué es un antecedente, qué es un consecuente... Se enseña también en este arte la solución de todos los argumentos sofísticos (es decir, aquellos argumentos lógicamente incorrectos pero psicológicamente persuasivos), y no es posible en ciencia alguna inferir sofísticamente lo falso de lo verdadero sin que, gracias a las reglas ciertas que esta ciencia da, se descubra con toda facilidad ese defecto, y, al contrario, sin ella, eso es imposible. Por eso, los que ignoran la Lógica, tienen por sofismas muchas demostraciones, y, a la inversa, estiman como sofismas a muchas demostraciones, por no saber distinguir entre el silogismo sofístico y el demostrativo. \par
    
    Otra utilidad de la Lógica es la facilidad en percibir el sentido del lenguaje y el modo propio de hablar. En efecto, con su ayuda se llega a conocer con facilidad qué es lo que los autores dicen y qué no, o qué se dice en sentido literal y qué en sentido metafórico; cosa muy necesaria a todos los que se dedican a leer a otros; pues los que toman siempre lo que dicen los autores en su sentido literal, vienen a dar en muchos errores y en dificultades insolubles. \par
    
    En cuarto lugar, hay que ver la diferencia y distinción de esta ciencia con las demás. Es de saber que la Lógica se distingue por sí misma de toda otra ciencia, porque su objeto es distinto del de las otras. En efecto, la Lógica proporciona, al menos principalmente, el conocimiento de los conceptos..." \par
    

\newpage

\subsection{``Metafísica''}
    
\begin{center}
    \large{\textbf{San Alberto Magno} [1193 - 1280]}
\end{center}
    Fue un sacerdote, obispo doctor de la Iglesia, destacado teólogo, geógrafo, filósofo y figura representativa de la química y en general, un polímata de la ciencia medieval. Perteneció a la orden dominica. Su humildad y pobreza fueron notables. \par
    
\begin{center}
    \large{\textbf{Libro II, Capítulos XII y XIII}}
\end{center}
    
    ``...los que están acostumbrados a los estudios filosóficos quieren en todo lo que oyen una certeza, o de evidencia inmediata, o de demostración. A otros, hechos a la vulgaridad y a la ignorancia, les parece triste y adusta la certeza filosófica, o porque, al no haber estudiado, no son capaces de entender tal lenguaje, ignorando la eficacia del silogismo, o por la cortedad o defecto de la razón o del ingenio. En efecto, una verdad que se obtiene con certeza por vía silogística, es de tal condición que no puede fácilmente alcanzarla el que no estudie, y está totalmente incapacitado para ella el que sea de cortos alcances... \par
    
    Los remedios contra los impedimentos señalados son: Contra el que proviene de la fuerza de la costumbre y lleva consigo la ignorancia de la mala disposición, hay que instruirle al hombre en el modo de pensar filosófico, con el cual sepa cómo hay que recibir lo que se dice... Ese modo de pensar se da en las ciencias lógicas en todas sus variedades de argumentaciones perfectas e imperfectas, porque con ellas estaremos capacitados para razonar silogísticamente sobre cualquier problema y, al mantener una disputa, no diremos nada inadmisible (...) ni desconoceremos la estructura de las pruebas y no aceptaremos como verdadero lo falso y lo no demostrado como demostrado. Por eso, los que no están formados en la Lógica, vemos que incurren en errores en todas las materias." \par
    

    
\subsection{``Comentario a los `Posteriores analíticos' de Aristóteles''}
    
    \begin{center}
        \large{\textbf{Santo Tomás de Aquino} [1225 - 1275]}
    \end{center}

    Fue un teólogo y filósofo católico perteneciente a la orden dominica. En materia de metafísica, su obra representa una de las fuentes más citadas del siglo XIII además de ser punto de referencia de las escuelas del pensamiento tomista y neotomista. La Iglesia católica lo nombra Doctor Angélico, Doctor Común y Doctor de la Humanidad y considera su obra fundamental para los estudios de filosofía y teología. \par
    
    \begin{center}
        \large{\textbf{Proemio}}
    \end{center}
    
    ``Como dice Aristóteles al comienzo de su `Metafísica' (Libro I, cap. I, n.3), el género humano vive por el arte y los razonamientos: En lo cual parece el filósofo alcanzar algo propio de los hombres, por lo que difiere de los demás animales. Efectivamente, los demás animales son llevados a sus actos por cierto instinto natural; en cambio, el hombre se dirige en sus acciones por el juicio de la razón. Y por eso es que, para que se realicen los actos humanos fácil y ordenadamente, diversas artes sirven (a tal propósito). En efecto, el arte no parece ser otra cosa que cierta ordenación de la razón acerca de qué modo, por medios determinados, los actos humanos lleguen a su debido fin. \par
    
    Mas la razón no sólo puede dirigir los actos de las partes inferiores (del hombre), sino que también es directiva de sus propios actos. Esto, en efecto, es propio de la parte intelectiva: El que reflexione sobre sí misma; por eso el intelecto se entiende a sí mismo y la razón puede razonar sobre su propio acto. Así como, por el hecho de que la razón raciocina (o razona) sobre el acto de las manos, se ha inventado el arte edificatorio o fabril, mediante los cuales el hombre puede ejercer esos actos fácil y ordenadamente, por la misma razón es necesario cierto arte que sea directivo del acto mismo de la razón, por el cual, a saber, el hombre pueda proceder ordenada, fácilmente y sin error en el mismo acto de la razón. \par
    
    Y tal arte es la Lógica, esto es, la ciencia racional. La cual no es racional solamente por el hecho de que proceda según la razón (lo que es común a todas las artes), sino también por esto: Porque recae sobre el acto mismo de la razón como a sobre su materia propia. \par
    
    Y, por lo tanto, parece ser (la Lógica) el arte de las artes, dado que nos dirige en el acto de la razón, de la cual proceden todas las artes. Es necesario, entonces, dividir a la Lógica en partes según la diversidad de los actos de la razón. \par
    
    Mas los actos de la razón son tres: De los cuales los dos primeros son actos de la razón en cuanto que ésta es cierto intelecto. \par
    
    Así, \emph{una primera acción} del intelecto es la intelección de los (objetos) indivisibles o  incomplejos, acción mediante la cual se concibe \emph{qué es} una cosa (...). Y a esta operación de la razón se refiere la doctrina que enseña Aristóteles [Filósofo griego 384-322 antes de Cristo] en su libro `De los predicamentos' (o `Categorías'). La \emph{segunda} operación del intelecto es la composición o división del intelecto, en la cual ya se da lo verdadero y lo falso. Y a este acto de la razón sirve la doctrina de Aristóteles en el libro `Sobre la interpretación'. El \emph{tercer} acto de la razón es aquel que es propiamente de la razón, a saber, el discurrir de una a otra cosa, de manera que a partir de lo que es conocido se llegue al conocimiento de lo desconocido. Y a este acto sirven `los demás libros de la Lógica'.'' \par
    
    *\textbf{Nota}: Por ``arte'', Santo Tomás no se refiere en estos pasajes principal ni solamente a las bellas artes, sino que le da un sentido más general, equivalente a técnica; pero no sólo a las técnicas o artes mecánicos, sino también a las llamadas artes liberales, como la Lógica, ,la Matemática y otras. \par
    
    Aquí nos detendremos para hacer una semblanza sobre la persona de Tomás de Aquino [1225-1275]. Nos olvidaremos por un momento de cronologías exactas, del número y detalles de sus obras, obviamente no porque no fueran importantes, sino para apuntar al hombre, al santo. \par
    
    Nacido en la ciudad de Nápoles, fue fraile de la Orden de Predicadores de Santo Domingo (u Orden de los Dominicos), fundada por el español Domingo de Guzmán en el 1215, en  Toulouse, Francia. \par
    
    Fray Tomás era de alta estatura (1,90m), recto, un tanto grueso, de frente amplia, de porte distinguido y de una sensibilidad extraordinaria. Cualquier cambio atmosférico o de clima le afectaba, y era sumamente sensible al frío. \par
    
    Su sobriedad era extrema. No comía ni bebía más que una sola vez al día (al mediodía), y no se preocupaba por lo que le ponían delante, y tenían que cuidar que tomase algo, porque se distraía continuamente en sus reflexiones. \par
    
    Fue muy amante de la pobreza. Cuando escribía la obra ``Suma contra Gentiles'' usaba cuadernillos de papel mediocre, aprovechándolos hasta la última línea y el último ángulo. Se contentaba con un hábito rustico de monje y un calzado muy pobre. En su habitación no se hallaba nada superfluo ni selecto. \par
    
    Su humildad fue extraordinaria. Jamás hablaba de sí mismo. Cuando fue atacado como Profesor en la Universidad de París, nunca se le escapó un gesto arrogante ni una palabra despectiva o molesta para nadie, tanto en público como en privado. Con ser un profesor tan célebre y admirado por muchos, jamás sintió el menor movimiento de vanidad o soberbia. \par
    
    Su inteligencia era rápida, profunda y equilibrada; prodigiosa su memoria; insaciable sus ansias de saber y su laboriosidad no conocía descanso. Comprendía con facilidad cuanto oía o leía, y lo retenía fielmente en su memoria como en el mejor fichero. Y es así que, a pedido de un fraile amigo, escribe una lista de 16 consejos para adquirir el tesoro de la ciencia. \par
    
    Estos son sus consejos: \par

\begin{enumerate}
    \item[``1.] Entra en el mar por los arroyos, no de una vez; porque es por medio de lo más fácil que conviene llegar a lo más difícil.
    \item[2.] Quiero que seas tardo para hablar, tardo para ir allí donde se habla.
    \item[3.] Conserva la pureza de conciencia.
    \item[4.] No ceses de entregarte a la reflexión.
    \item[5.] Frecuenta con amor tu ámbito de estudio.
    \item[6.] Sé amable para con todos.
    \item[7.] No te pongas a averiguar nada de los hechos ajenos.
    \item[8.] No seas demasiado familiar con nadie, pues el exceso de familiaridad puede dar motivo para que te desprecien y puede dar ocasión de abandonar el estudio.
    \item[9.] No intervengas en modo alguno en palabras o cosas que no conduzcan a nada.
    \item[10.] Huye por sobretodo de las gestiones inútiles.
    \item[11.] Imita la conducta de los santos y de los hombres de bien.
    \item[12.] No mires a quien escuchas, pero lo que diga de bueno has de tenerlo siempre presente.
    \item[13.] Trata de comprender aquello que leas y oigas.
    \item[14.] Preocúpate de aclarar tus ideas.
    \item[15.] Esfuérzate por ordenar todo lo que puedas en tu mente, como se llena un vaso.
    \item[16.] No busques aquello que te sobrepasa.
\end{enumerate}
    
    Si sigues estas huellas, darás y llevarás, durante el tiempo de tu vida, hojas y frutos útiles en la viña del Señor. Si te sujetas a estos consejos podrás alcanzar lo que deseas'' \par

    Sigamos, pues, estas huellas recordando que la etimología de la palabra \emph{investigación} proviene del latín 'in'=en, dentro y de 'vestigio'=huella. Investigar no es otra cosa que estar en la huella de algo, es un proceso de búsqueda de la verdad a través de sus vestigios o huellas. Y para esto necesitamos tener una mente ordenada por la Lógica. \par 
        
        
\newpage

    
\subsection{``Introducción a la Lógica''}
    
    \begin{center}
        \large{\textbf{Irving Copi} [1917 - 2002]}
    \end{center}

    Fue un filósofo, lógico estadounidense, y autor de varios textos universitarios. Alcanzó la fama luego de publicar Introducción a la lógica (Introduction to Logic) y La lógica informal (Informal Logic). Ambas obras se utilizan en la actualidad. \par
    
    \begin{center}
        \large{\textbf{Prefacio}}
    \end{center}
    
    ``El estudio de la lógica ofrece obvios beneficios: Mayor capacidad para expresar ideas con claridad y concisión; aumentar en la habilidad para definir los propios términos; enriquecimiento de la capacidad para formular razonamientos con rigor y examinarlos críticamente. Pero su mayor provecho, a mi juicio, reside en el reconocimiento de que la razón puede ser aplicada a todo aspecto de los asuntos humanos. \par
    
    Las instituciones democráticas son atacadas hoy desde todos los flancos. La mejor manera de defenderlas es hacerlas funcionar. Y sólo esto se logra si cada ciudadano piensa en sí mismo, si discute libremente con sus semejantes, si delibera, si evalúa los elementos del juicio y reconoce que, con un poco de esfuerzo, podemos establecer la diferencia entre los buenos y malos razonamientos. Si queremos gobernarnos bien y de manera responsable, debemos ser razonables. El estudio de la lógica puede brindarnos, no sólo la práctica del razonamiento, sino el respeto por la razón.'' \par
    
    \begin{center}
        \large{\textbf{Capítulo I - Introducción \\
        ¿Qué es la lógica?}}
    \end{center}
    
    ``\emph{La lógica es el estudio de los métodos y principios usados para distinguir el buen (correcto) razonamiento del malo (incorrecto).} No debe interpretarse esta definición en el sentido de que sólo el estudioso de la lógica puede razonar bien o correctamente. [...] \par
    
    ...la persona que ha estudiado lógica tiene mayor probabilidad de razonar correctamente que aquella que nunca ha pensado en los principios generales implicados en esta actividad. Ello se debe a varias razones. Ante todo un estudio adecuado de la lógica la enfocará tanto como un arte como una ciencia, y el estudiante deberá hacer ejercicios concernientes a todos los aspectos de la teoría que aprende. Aquí, como en todo, \emph{la práctica ayuda a perfeccionarse}. En segundo lugar, una parte tradicional del estudio de la lógica consiste en el examen y el análisis de los métodos incorrectos de razonamiento, o sea las falacias. Esta parte de la materia no sólo da una visión más profunda de los principios del razonamiento en general, sino que el conocimiento de esas trampas nos ayuda positivamente a evitarlas. Por último, el estudio de la lógica suministrará al estudiante ciertas técnicas y ciertos métodos de fácil aplicación para determinar la corrección o incorrección de muchos tipos diferentes de razonamientos, incluso los propios. Y cuando es posible localizar fácilmente los errores, es menor la posibilidad de que se cometa. \par
    
    \emph{La lógica ha sido determinada a menudo como la ciencia de las leyes del pensamiento.} Pero esta definición, aunque ofrece un indicio de la naturaleza de la lógica, no es exacta. En primer lugar, el pensamiento es uno de los procesos estudiados por los psicólogos. La lógica no puede ser `la' ciencia de las leyes del pensamiento porque también la psicología es una ciencia que trata las leyes del pensamiento (entre otras cosas). Y \emph{la lógica no es una rama de la psicología; es un campo de estudio \textbf{separado y distinto}.} \par
    
    [...] Otra definición de lógica es aquella que la considera como \emph{la ciencia del razonamiento.} Esta definición es mejor, pero no es aún adecuada. El razonamiento es un tipo especial de razonamiento en el cual se realizan inferencias, o sea en el que se derivan conclusiones a partir de premisas. \textbf{Pero} es aún un tipo de pensamiento, y por lo tanto forma parte del tema de estudio del psicólogo. \par
    
    [...] \emph{La distinción entre el razonamiento correcto e incorrecto es el problema central que debe tratar la lógica.} Los métodos y las técnicas del lógico han sido desarrollados especialmente con el propósito de aclarar esta distinción. El lógico se interesa por todos los razonamientos, sin tomar en cuenta su contenido, pero solamente desde este especial punto de vista.'' \par
    

\subsection{``Introducción a la Filosofía''}
    
    \begin{center}
        \large{\textbf{Héctor Mandrioni} [1920 - 2010]}
    \end{center}
    
    Fue un sacerdote y filósofo argentino contemporáneo. Su actuación más conocida fue en el campo de la filosofía. Doctorado en la Universidad Nacional de La Plata, amplió y profundizó sus estudios en las universidades de Munich, Tübingen, Heidelberg y Freiburg. Dictó cátedra en distintas universidades y profesorados argentinos. Autor de numerosos libros. \par
    
    \begin{center}
        \large{\textbf{Capítulo 6 Párrafo 2 \\
        El conocimiento de las disciplinas filosóficas }}
    \end{center}
    
    ``El aspecto meramente formal del conocimiento abordado por la lógica, que \emph{estudia las leyes que rigen el desarrollo formal del proceso cognoscitivo racional}. La lógica ha sido definida, como la \emph{disciplina que enseña las reglas por medio de las cuales la razón humana puede adquirir con orden, facilidad y sin error el conjunto de las ciencias.} Eminentemente reflexivo, el saber lógico, implica un retorno a la razón sobre sí misma, sobre sus propios actos vitales, captándose en su propio actuar a fin de poder determinar las leyes que regulan su recto operar. No le interesa a la lógica, la realidad o las cosas en sí mismas, alcanzadas por la mente, sino \emph{el recto funcionar de la razón para conocer sin error las cosas}. Es un estudio estrictamente formal de las funciones mentales, a saber, simple aprehensión, juicio y raciocinio, independientemente del valor real del contenido de esas operaciones.'' \par
    
    
    \textbf{Nota}: Lo resaltado no es por parte de los autores. \par

\vspace*{\fill}
    
\noindent Lecturas sugeridas:
\begin{itemize}[label={$\bullet$}]
    \item Maritain, J. - Introducción a la filosofía. Ed. Club de Lectores, Bs. As., 1984. Párrafos 29 a 37.
    \item Maritain, J. - El orden de los conceptos. Ed. Club de Lectores, Bs. As., 1980. Párrafos 1 a 4.
    \item Casaubón, J. - Nociones generales de lógica y filosofía. Ed. Estrada, Bs. As., 1981. Unidades 1 y 2.
    \item Sanguinetti, J. - Lógica. EUNSA, Pamplona, 1985. Parte introductoria. Capítulos 1 y 2.
    \item Millán Puelles, J. - Fundamentos de filosofía. Rialp, Madrid, 1981. Primera parte. Capítulo 3.
    \item Copi, I. - Introducción a la lógica. EUDEBA, Bs. As., 1990. Capítulo 1. Introducción. 
\end{itemize}
    
\newpage
    
    \noindent Analizar los 5 textos arriba citados.
    
    \noindent Realizar la siguiente actividad interdisciplinaria. \\
    
    \noindent \textbf{Actividad interdisciplinaria} \\
    
    \noindent Realizaremos el siguiente ejercicio: \\
    \indent a.) Traducir al castellano. \\
    \indent b.) Trazar una línea y dividirla ``por siglos'' (desde el siglo V antes de Cristo hasta el siglo XX), luego ubicar a todos los filósofos y lógicos citados en el texto. 
    
    
    
\begin{center}
        \large{\textbf{Richard H. Popkin} [1923 - 2005]}
    \end{center}

    Fue un filósofo académico que se especializó en la historia de la filosofía de la ilustración y en los primeros anti-dogmatismos modernos. \par
    
    \begin{center}
        \large\textbf{{Logic}} 
    \end{center}
    
    Logic is the systematic study of reasoning that provides standards by which valid reasoning can be recognized. It clarifies the reasoning process and provides a means for analyzing the consistency of basic concepts. Logic has played an important role in the history of Philosophy. \par
    
    \begin{center}
        \large\textbf{{Traditional logic}} 
    \end{center}
    
    The history of Western logic can be traced to ancient Greece. Logic had developed independently in China and India but apparently had little influence in the West. Aristotle (384 - 322 BC) is usually considered the first major Western logician, although earlier contributions to logic were made by Plato (429/427 - 348/347 BC), Socrates (470/469 - 399 BC), Zeno of Elea (490/485 - ? BC), and others. Aristotle's logical system treated the categorial Syllogism and the laws of logic. His many books became the basis for its study up to the 19th century. The next school of logicians, the Megarians, flourished in the 4th century BC. Rather than study categorical inferences, they sought to define the conditions under which a conditional, ``If A, then B'', is true. The Stoics, especially one of their leaders Chrysippus (281 - 204 BC), took over and developed the logical ideas of the Megarians (can see Stoicism). Aristotle had set forth a logic of terms: Chrysippus worked out a logic of propositions. Stoic thinkers after Chrysippus apparently did not contribute further to development of logic. Other late Greek and Roman logicians mainly codified the work of their predecessors, although Galen (129 - 199 AD) added some theories about special kinds of syllogisms. \par
    
    The logic of the Greeks endured in the Middle East after the fall of Rome and the conquest of Western Europe by the barbarians tribes. In the Middle Ages logic was again brought to the attention of the Western world by logicians of the Islamic empire, mainly those working in Baghdad and southern Spain. Many of Aristotle's writings as well as commentaries preserved by heretical Christian sects were available to these logicians. Al-Farabi (870 - 950), one of the greatest Muslim logicians, wrote commentaries on most of Aristotle's logical works. Other logicians translated other Greek works on logic into Arabia. (From Arabic they were usually translated by Jewish scholars into Hebrew, and then from Hebrew into Latin: In this way these texts became available to scholars in Christian Europe). The great Muslim philosopher Avicenna (979 - 1037) made logic independent of the teachings of Aristotle and the Stoics. By the 14th century, intellectuals were reading handbooks on logic by various muslim scholars, instead of the classics. \par
    
    Meanwhile, the logical materials that were being studied in the Islamic empire became known in Christian Europe. The Christians previously had only a few of Aristotle's works and a few commentaries. The first significant logician of the Christian Middle Ages, Peter Abelard (1079 - 1142), wrote before most of the Aristotelian materials became available, but developed detailed and critical evaluations of the material that had been preserved in the West. The rest of Aristotle's logical wirtings became available by about 1200, resulting in the emergence of the logica moderna, a ``new logic'' taught mainly in the arts faculties of the universities rather than in the theological schools. Perhaps the greatest of the modern logicians was William of Occam (1285? - 1350), who wrote the Summa Logicae (1326?). Others logicians restructured the domain of logic so that the Aristotelian heritage into a broader logic. \par
    
    As the Renaissance began, so did an attack on medieval (Scholastic) thought and on the Aristotelian theories that provided its basis. The major Renaissance opponent of Aristotelian logic was the 16th century French Protestant thinker Petrus Ramus (1515 - 1572), who is supposed to have debated ``that every proposition in Aristotle is false''. Ramus attacked nearly all of Aristotle's logical doctrines and proposed instead that there be a logic of invention or discovery and a logic of judgment. \par
    
    With the rejection of Aristotle's metaphysics by the great 17th century thinkers Francis Bacon (1561 - 1626), Rene Descartes (1596 - 1650), Baruch Spinoza (1632 - 1677), and John Locke (1632 - 1704), a search began for a new logic that would fit with a new picture of reality. Gottfried Wilhelm von Leibniz (1646 - 1716) made major contributions to this new logic, although most of his logical work did not become generally available until the end of the 19th century. Leibniz tried to work out a universal logical language, and he also developed a logical calculus. He was apparently not ready to reject Aristotelian logic, but his examinations of other possibilities when discovered at the end of the 19th century nevertheless alded in the development of modern logic. \par
    
    %\noindent Richard H. Popkin [Grolier Electronic Publishing, Inc. 1993]
    
\newpage
    
\section{Textos del capítulo 2}
    
\subsection{``Introducción a la Filosofía''}
    
\begin{center}
    \large{\textbf{Jacques Maritain} [1882 - 1973]} 
\end{center}
    Fue un filósofo católico francés, principal exponente del humanismo cristiano. Uno de los más destacados defensores del neotomismo, a partir del cual se propuso edificar una metafísica cristiana, que él denominó ``filosofía de la inteligencia y del existir''. \par

\begin{center}
    \large{\textbf{Capitulo II, Parte II, Sección I, ``Lógica'', párrafo 37}}
\end{center}
    
    ``Y ahora se presenta inmediatamente este problema: Siendo individuales y singulares todos los seres que existen en la naturaleza, \emph{¿Cómo puede ser verdadero el conocimiento que adquirimos por medio de nuestros conceptos, que solo y siempre nos dan lo universal?} \par
    
    Esta cuestión, que nos obliga a inquirir con toda diligencia en qué consiste exactamente la universalidad del contenido de nuestros conceptos, no es en sí misma, pero sí para nosotros, \emph{el primero y más grave de los problemas de la Filosofía}. Se refiere, en efecto, a la inteligencia misma y a los conceptos, es decir, el instrumento mediante el cual adquirimos todos nuestros conocimientos; y la solución que a ella dan las diversas filosofías es la que las orienta en todas las demás cuestiones. \par
    
    De acuerdo con esta solución se puede agrupar a los filósofos en tres grandes escuelas: \par

\begin{center}
    \textbf{1\textsuperscript{a} - Nominalismo}
\end{center}

    La escuela nominalista, para la cual \emph{el universal consiste nada más que en nombres o a lo más en conceptos, sin que tenga fundamento alguno en la realidad de las cosas} (por ejemplo, no existe en la realidad, ni es ninguna cosa real ``una naturaleza humana'' que se encuentre en Pedro, en Juan o en Andrés); doctrina que  \emph{\textbf{destruye} pura y simplemente el conocimiento intelectual}, y hace de la ciencia una ficción. Esta escuela tiene como representantes más caracterizados, en la antig\"uedad, a los sofistas y a los escépticos, y en los tiempos modernos a los maestros de filosofía inglesa: Guillermo de Ockham en el siglo XIV, Robbes y Locke en el siglo XVII, Berkeley y Hume en el siglo XVIII, Stuart Mill y Spencer en el siglo XIX. Se han de tener presente que la mayor parte de los filósofos ``modernos'' (es decir, que ignoran o son adversarios de la tradición escolástica) están más o menos conscientemente contagiados del nominalismo. \par 

\begin{center}
    \textbf{2\textsuperscript{a} - Realismo Absoluto}
\end{center}

   La escuela realista absoluta: Para ella el universal como tal, el universal independiente de los seres, tal como existe en la mente, constituye la realidad de las cosas: Con lo que el conocimiento sensitivo se reduce a una ilusión. La única realidad es, por ejemplo, una ``naturaleza humana'' que existe *fuera de la mente, en sí misma, separada de todos los seres*; un ``hombre en sí'' (como en el sistema de Platón); o lo que es más, un ``Ser común'', existiendo como tal, *fuera de la mente* como una sola y única sustancia (como en la doctrina Parménides y en la filosofía brahamánica). Algunos filósofos modernos (Spinoza, Hegel) se acercan más o menos al realismo absoluto. \par

\newpage

\begin{center}
    \textbf{3\textsuperscript{a} - Realismo Moderado}
\end{center}

    La escuela que profesa el realismo moderado se trata de una doctrina verdaderamente original, que guarda \emph{el justo medio} entre el nominalismo y el realismo absoluto, gracias a una visión más elevada de las cosas, y no a una atenuación del realismo absoluto. Esta escuela, haciendo distinción entre la ``cosa'' y el ``modo de existir'' de esa cosa misma, enseña que \emph{las cosas están en la mente de un modo universal, y en la realidad de un modo individual}. Por consiguiente aquello que nosotros percibimos por medio de nuestros conceptos, en un estado de universalidad, existe realmente, pero en las cosas mismas y por lo tanto en un estado de individualidad. Así por ejemplo, hay en la realidad una ``naturaleza humana'' que se encuentra en Pedro así como en Pablo, etc., pero, fuera del espíritu, existe en esos sujetos individuales, e identificada con cada uno de ellos, y no en sí misma, separada de esos seres. El realismo moderado es la doctrina de Aristóteles y de Santo Tomás de Aquino.'' \par
    
\newpage ,m
    
\subsection{``Sócrates''}
    
\begin{center}
    \large{\textbf{Rodolfo Mondolfo} [1877 - 1976]} 
\end{center}
    Nació el 20 de agosto de 1877 en Senigallia, Italia. Emigró a Argentina y enseñó en las universidades de Córdoba y Tucumán. Uno de los aspectos más interesantes de su obra son los estudios sobre la filosofía antigua, la filosofía del renacimiento y la filosofía marxista. Una de sus más significativas contribuciones fue completar la importante historia de la filosofía griega de Edward Zeller, en una notable edición. \par 
    
\begin{center}
    \large{\textbf{Capítulo 7: La ciencia y los conceptos universales}}
\end{center}
    
    ``Su investigación no quería versar en lo audable (objeto solamente de opinión) sino  en lo inmutable, es decir, \emph{lo universal, la esencia} (objeto de la ciencia). Como decía Aristóteles (``Metafísica'' Libro I, IV, 987), `Sócrates discutía solamente acerca de las cosas morales y no se interesaba en absoluto en la naturaleza; y en las cosas morales \emph{buscaba lo universal}, pues fue el primero que tomó como objeto de su pensamiento las definiciones'. Y agrega Aristóteles en otro lugar de la ``Metafísica'' (Libro XIII, Capítulo IV, 1078): `Tenía razón en buscar las esencias (lo que es cada cosa) pues quería razonar, y el principio de los razonamientos está constituído por la esencia de las cosas'. \par
    
    (...) Así se tiene la unidad del concepto a través de la multiplicidad de los sujetos y de las inteligencias; pero junto con esa unidad subjetiva debe buscarse y lograrse también la objetividad, vale decir, la unidad del concepto a través de la multiplicidad de las cosas y de los hechos. Esta doble unidad conjunta es lo que busca la ciencia. \par

        \setlength{\leftskip}{3em}
        \noindent `Yo buscaba -dice Sócrates, en una obra de Platón, en ``Menón'' 72 y sigs.- una única virtud y encuentro ahora un enjambre, si te pregunto, ¿cuál es la naturaleza de las abejas?, contestarás que hay muchas abejas y de muchas especies. Pero... si te pregunto, ¿qué es aquello por lo cual las abejas no son distintas sino que son todas abejas?... Y, en lo que respecta a las virtudes... Si alguien preguntase, ¿qué es la figura?... sin duda encontraríamos muchas figuras distintas; pero no es esto lo que quiero, sino que, puesto que a todas, a pesar de su oposición recíproca, las llamo figuras... quiero saber lo siguiente: ¿qué es lo que llamas figuras? ¿No entiendes que \emph{busco lo que hay de igual} en lo redondo, en lo recto y en otras figuras?'. \par
    
    \setlength{\leftskip}{0em}
    Así se perfila \emph{el camino de la ciencia en el paso de la multiplicidad de los particulares a la unidad de lo universal} por la \emph{inducción} y en la \emph{determinación exacta de ese universal} por la \emph{definición}, los dos elementos del método científico, cuyo mérito según Aristóteles (``Metafísica'' Libro XIII, Capítulo IV, 1078) corresponde a Sócrates.'' \par
    
\vspace*{\fill}
    
\noindent Lecturas sugeridas:
\begin{itemize}[label={$\bullet$}]
    \item Casaubón, J. - Op. cit. Cap. 3
    \item Maritain, J. - El orden de los conceptos. Op. cit. Cap. 1
    \item Millán Puelles, J. - Op. cit. Cap. 4
    \item Sanguinetti, J. - Op. cit. 1\textsuperscript{a} parte, Caps. 1 y 2
    \item Ferrater Mora, J. - Diccionario de Filosofía. [Ver el término `universal']
    \item Copi, I. - Op. cit. [Ver 1\textsuperscript{a} parte, Cap. 4 `La definición']
\end{itemize}
    
\newpage
    
\section{Textos del capítulo 3}

\subsection{``Memorias''}

\begin{center}
    \large{\textbf{John Stuart Mill} [1806 - 1873]} 
\end{center}
    Fue un filósofo, político y economista inglés de origen escocés. En 1843, a sus 37 años de edad publicó su primer libro que le costó trece años en escribirlo, Un sistema de lógica (``System of logic''). En él, manifiesta su más extremo del empirismo epistemológico, basado también en sus pensamientos liberales. John Stuart Mill discute el propósito de la lógica en la comprensión humana. \par 

\begin{center}
    \large{\textbf{(Fragmento tomado de Maritain ``El orden de los conceptos'', prólogo)}}
\end{center}
    
    ``Estoy convencido, a propósito de la Lógica, de que nada contribuye más, cuando se hace un uso juicioso de ella, a formar pensadores exactos, fieles al sentido de las palabras y de las proposiciones, y a poner en guardia contra los términos vagos, poco apropiados y ambiguos. Se alaba mucho el estudio de las Matemáticas para alcanzar este resultado: no es nada en comparación con el estudio de la Lógica. En efecto, en las operaciones matemáticas no encontramos ninguna de las dificultades que son los obstáculos de un razonamiento incorrecto (por ejemplo en Matemáticas, las proposiciones sólo son universales afirmativas; por lo demás, los dos términos están reunidos por el signo igual [``$=$''], de donde la posibilidad inmediata de la conversión pura y simple, etc.)\ldots'' \par
    Por otra parte, muchos hombres capaces no logran desenmarañar una idea confusa y contradictoria, por no haber estado sometido a esta disciplina\ldots'' \par
    
    \noindent Completar la lectura de este texto determinando el valor de verdad de las siguientes proposiciones: \par
    
\begin{center}
\begin{tabular}{c l c l}
    A & - Todo $2 + 2 = 4$ (V)  & \hspace{2cm} E & - Ningún $2 + 2 = 4$ (F)        \\
    E & -                       & \hspace{2cm} A & -                               \\
    I & -                       & \hspace{2cm} I & -                               \\
    O & -                       & \hspace{2cm} O & -                               \\
                                &                                              \\
    I & - Algún $2 + 2 = 4$ (V) & \hspace{2cm} O & - Algún $2 + 2$ no es $= 4$ (F) \\
    A & -                       & \hspace{2cm} A & -                               \\
    E & -                       & \hspace{2cm} E & -                               \\
    O & -                       & \hspace{2cm} I & -                               \\
\end{tabular}
\end{center}

\noindent ¿Qué tipo de proposición parece más útil a las Matemáticas y por qué?

\newpage

\subsection{``Suma Teológica''}

\begin{center}
    \large{\textbf{Santo Tomás de Aquino} [1225 - 1275]}
\end{center}
    Fue un teólogo y filósofo católico perteneciente a la orden dominica. En materia de metafísica, su obra representa una de las fuentes más citadas del siglo XIII además de ser punto de referencia de las escuelas del pensamiento tomista y neotomista. La Iglesia católica lo nombra Doctor Angélico, Doctor Común y Doctor de la Humanidad y considera su obra fundamental para los estudios de filosofía y teología. \par
    
    \begin{center}
        \large{\textbf{I Parte, cuestión 85, artículo 5, ``corpus'' }}
    \end{center}
    
    ``Debe decirse que el entendimiento humano necesita entender componiendo y dividiendo: porque, como pasa de la potencia al acto, tiene cierta analogía con los seres generales, que no tienen de inmediato su perfección, sino que la adquieren paulatinamente. Del mismo modo el intelecto humano no adquiere en su primera aprehensión el conocimiento perfecto del objeto, sino que aprehende primeramente algo de él, como el ``quid'' de la cosa misma, que es el primero y el propio objeto del intelecto, y después distingue las propiedades y accidentes y demás condiciones circunstanciales de su esencia. Y según esto es necesario componer y dividir los objetos aprehendidos unos con otros, y proceder de una composición y división a otra, lo cual es raciocinar.\par
    Pero el intelecto del ángel y el de Dios son como las cosas incorruptibles, que llegan al apogeo de su perfección desde el principio mismo de su existencia; y por lo tanto el entendimiento del ángel y el de Dios tienen todo el conocimiento perfecto de las cosas; y por consiguiente, al conocer el ``quid'' del objeto conocen al propio tiempo cuanto de él podemos conocer nosotros componiendo, dividiendo y raciocinando\ldots'' \par
    
    \noindent ¿Qué diferencia existe entre el entendimiento angélico y el entendimiento divino con respecto al humano? \par

\subsection{``El verbo abraza la idea del tiempo''}

\begin{center}
    \large{\textbf{Aristóteles} [384 - 322 a. C.]}
\end{center}
    Fue un filósofo, polímata y científico nacido en la ciudad de Estagira, al norte de la Antigua Grecia. Aristóteles escribió cerca de 200 tratados (de los cuales solo se han conservado 31) sobre una enorme variedad de temas, entre ellos: lógica, metafísica, filosofía de la ciencia, ética, filosofía política, estética, retórica, física, astronomía y biología. Aristóteles transformó muchas, si no todas, las áreas del conocimiento que abordó. Es reconocido como el padre fundador de la lógica y de la biología. \par
    
\begin{center}
    \large{\textbf{Sobre la interpretación \\
    Capítulo IX: Sobre el verbo o cópula \\
    Parágrafos 1 a 6}}
\end{center}
    
\begin{enumerate}
    \item El verbo es la palabra que, además de su significación propia, abraza la idea de tiempo, y ninguna de sus partes tomada aisladamente tiene sentido por sí misma, siendo siempre el signo de las cosas atribuidas a otras cosas.
    \item Digo que abraza la idea de tiempo además de su significación propia, porque, por ejemplo, ``la salud'', no es más que un nombre; ``está sano'' es un verbo, porque expresa además que la cosa se verifica en el momento actual.
    \item Al propio tiempo es siempre el signo de las cosas atribuidas a otras cosas; como por ejemplo, cuando se dicen cosas de un sujeto o que están en un sujeto.
    \item ``No está sano'', ``no está enfermo'', no son en mi opinión verbos, no obstante indican tiempo y se refieren necesariamente a algo. Pero esta diferencia no ha recibido ningún nombre en especial; le llamaremos, si se quiere, un verbo indeterminado, porque puede aplicarse a todo, tanto al ser como al no-ser.
    \item De la misma manera `` ha estado sano'' y ``estará sano'', no son propiamente verbos, sino que son tiempos del verbo, difieren de éste en que el verbo indica el tiempo presente, mientras que los otros son  tiempos accesorios (lo pasado y lo futuro).
    \item Los verbos tomados aisladamente y en sí mismos son nombres y significan un objeto especial (Nota del traductor y comentador Patricio de Azcárate: ``correr'' en sí y como nombre, sólo expresa, sin la adición de tiempo y de los modos, la idea de ``carrera'', y ni la afirma, ni la niega. Por sí mismo no tiene un sentido completo. El verbo ser está en el mismo caso. Es preciso `predicar' algo que se una al sujeto para que el pensamiento sea completo) al pronunciarlos se fija el pensamiento del que oye, y se detiene su espíritu. Pero nada hay todavía que exprese que la cosa es o no es. Ser o no ser no es el signo de la cosa misma, como no lo es si se expresa el ser en sí y completamente aislado. El verbo ser por sí sólo es nada; sólo indica, además de su sentido propio, cierta combinación, que de ninguna manera puede comprenderse independientemente de las cosas que lo forman.'' 
\end{enumerate}

\noindent ¿Qué relación existe, según Aristóteles, entre el verbo y el tiempo?
    
\vspace*{\fill}

\noindent Lecturas sugeridas:
\begin{itemize}[label={$\bullet$}]
        \item Casaubón, J. - Op. cit. Cap. 3
        \item Maritain, J. - El orden de los conceptos. Op. cit. Cap. 2
        \item Sanguinetti, J. - Op. cit. 2\textsuperscript{a} parte.
        \item Millán Puelles, J. - Op. cit. Cap. 5
        \item Copi, I. - Op. cit. Cap. 5
\end{itemize}

\newpage

\section{Textos del capítulo 4}

\end{document}
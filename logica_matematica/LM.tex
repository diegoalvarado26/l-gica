\documentclass{article}

\usepackage[utf8]{inputenc}
\usepackage[T1]{fontenc}
\usepackage{imakeidx}
\usepackage[spanish]{babel}
\setlength{\parindent}{2em}
\setlength{\parskip}{1em}
\renewcommand{\baselinestretch}{1.1}

\usepackage{enumitem} % para utilizar [noitemsep, nolistsep]
\setlist[enumerate]{label*=\arabic*.} % te crea automaticamente sub enumeraciones 

\usepackage{schemata}
\newcommand\diagram[2]{\schema{\schemabox{#1}}{\schemabox{#2}}}
\begin{document}

\tableofcontents

\newpage

\section{La Ubicación de la Lógica Proposicional}
Ubicaremos a la Lógica Proposicional como una parte de la Lógica Matemática o Simbólica. 
\\

\diagram{Lógica}{ \diagram{Clásica o Filosófica}{Concepto \\ Enunciación \\ Argumentación}\smallskip
\\ \diagram{Matemática o Simbólica} {Proposicional \\ \diagram{De Términos}{ de Predicados \\ de Clases \\
de Relaciones}\smallskip}\smallskip}\smallskip

\section{Proposiciones Atómicas y Moleculares}
Esta lógica toma el nombre de proposicional porque trabaja exclusivamente con proposiciones. Recordemos que la proposición se caracteriza por afirmar o negar algo de algo y de tener la propiedad de ser V (verdadera) o F (falsa).

Estas proposiciones \textbf{se conectan} entre sí por medio de determinadas \textbf{locuciones} que se simbolizarán así:

\begin{center} \bfseries
     \begin{tabular}{| c | c |} 
     \hline
     LOCUCIONES & SÍMBOLO \\ 
     \hline
     No; no es cierto que; no es verdad que; etc & -\\ 
     \hline
     Y; pero; aunque; sino; etc & . \\
     \hline
     O; y/o; a menos que; etc. & \lor \\
     \hline
     O; o bien; etc. & \normalfont w \\
     \hline
     Si... entonces; es condición suficiente para...; etc.
     & \supset \\
     \hline
     Si, y solo si; ... es condición necesaria y suficiente para...; etc. & \equiv \\ 
     \hline
     Ni... ni... & \downarrow \\
     \hline
     ... es incompatible.. & / \\
     \hline
    \end{tabular}
\end{center}


\end{document}